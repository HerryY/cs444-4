\documentclass[a4paper]{article}

\usepackage[english]{babel}
\usepackage[utf8x]{inputenc}
\usepackage{amsmath}
\usepackage{graphicx}
\usepackage{bibentry}

\title{Week 3 Summary}
\author{Garrett Amidon}

\begin{document}
\maketitle




\paragraph{}

Chapter 14 of “Linux Kernel Development 3rd edition” (published June 2010), by Robert Love discussed what a block device was, different types of devices, the hardware that makes it up, and how the block device is used to make up a bio. The author starts off by defining what a block is before moving into different types of blocks and how they can be used and how they are implemented. By using the strategy of definition to implementation, the author is able to use examples of different hardware that uses a block and thus further help the reader understand how they are used. The audience this section is most shaped towards are people who understand code, such-as on page 294 where he defines what a bio is using C, and those who understand how a buffer works and can follow figures. 






\end{document}